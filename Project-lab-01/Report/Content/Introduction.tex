\section{Giới thiệu}
%--------------------------
Trong xã hội hiện tại hình ảnh kỹ thuật số rất cần thiết trong cuộc sống, từ việc lưu trữ kỷ niệm, nghiên cứu, nghệ thuật. Tuy nhiên, các tệp hình ảnh có độ phân giải cao thường có kích thước lớn và điều này sẽ dẫn đến việc xử lý, truyền tải và lưu trữ kém hiệu quả về mặt thời gian và không giam lưu trữ. Đó cũng là mục tiêu của kỹ thuật nén ảnh nhằm làm giảm kích thước ảnh mà không làm giảm chất lượng hình ảnh.

Báo cáo này thảo luận về việc áp dụng phương pháp nén màu K-mean nhằm làm giảm không gian lưu trữ cần thiết cho một hình ảnh. K-mean là một phương pháp học máy không giám sát, phân cụm các điểm dữ liệu dựa trên sự tương đồng. Kỹ thuật K-mean có thể làm giảm số lượng màu hiển thị trong hình ảnh khi nén hình ảnh và phương pháp này cung cấp sự cần bằng giữa độ trung thực của hình ảnh và tỷ lệ nén.

Các phần sau được tổ chức như sau, Phần 3 với các ý tưởng và mô tả về thuật toán K-mean và các hàm tiện ích. Phần 3 đưa ra kết quả với số lượng màu sắc cụ thể và nhận xét về các kết quả này. Các kết luận sẽ được đưa ra ở Phần 4.